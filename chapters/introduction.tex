% !TEX root = ../thesis.tex

\chaptermark{Úvod}
\phantomsection
\addcontentsline{toc}{chapter}{Úvod}

\chapter*{Úvod}

Napriek tomu že umelá inteligencia získala v posledných rokoch veľkú popularitu, tak sa nejedná o žiaden nový pojem. O umelej inteligencií sa písalo už v roku 1950. Vďaka technologickému pokroku však v posledných rokoch zaznamenala najväčší pokrok. A budúcnosť umelej inteligencie môže byť viac než pozoruhodná. 
\par Na základe veľkého pokroku v umelej inteligencii, vieme, že umelá inteligencia dokáže napodobniť ľudské činnosti. Už dnes môžeme povedať že umelá inteligencia dokáže prakticky vnímať, plánovať, riešiť problémy, myslieť, rozumieť, učiť sa. Tieto vlastnosti umelej inteligencie sú základným pilierom pre jej ďalší vývoj smerom k budúcnosti.
\par Umelá inteligencia a digitálna transformácia sú dotýkajú vo väčšej miere dokonca našich moderných životov. Toto zrýchlené tempo v súčasnej dobe poháňa umelú inteligenciu na vyššiu úroveň. Už dnes máme dostatočné množstvo AI online nástrojov, ktoré nám uľahčujú nielen každodenný život ale i programovanie.
 Umelá inteligencia otvára nové možnosti a zrýchluje pokrok v širokom spektre odvetví.
\par V diplomovej práci sa zameriavame na teoretické poznatky o umelej inteligencii, jej štruktúre, AI online nástrojoch, ich úlohe v programovaní, prekladu programovacích jazykov a pohľadu do budúcnosti umelej inteligencii. 
\par V druhej časti diplomovej práce zanalyzujeme dostupné online AI nástroje a preskúmame potenciál AI online nástrojov na použitie pri preklade programovacích jazykov. V analytickej časti diplomovej práci sa zameriavame najmä na faktory akými sú presnosť prekladu, podpora programovacích jazykov, dopad na výkon a riešenie prekladu špecifických jazykových konštrukcií, ktoré iné jazyky priamo nepodporujú. V analytickej časti vytvoríme vlastné experimenty, čím sme schopní vyhodnotiť využitie AI online nástrojov pri preklade programovacích jazykov.

Uvod sa ešte pozmení a doplní keď bude hotová celá DP



\section*{Formulácia úlohy}
Cieľom diplomovej práce je analyzovať dostupné AI online nástroje a preskúmanie ich potenciálu na použitie pri preklade programovacích jazykov. Je potrebné zamerať sa na hlavné faktory ktorými sú:
\begin{itemize}
\item presnosť prekladu
\item podpora programovacích jazykov
\item efektivita kódu
\item riešenie prekladu špecifických jazykových konštrukcií, ktoré iné jazyky priamo nepodporujú
\end{itemize}

Po oboznámení sa zo zadaním úlohy sme vyhodnotili že riešenie bude nasledovné:
\begin{itemize}
\item Vo všeobecnosti sa oboznámime s pojmom umelá inteligencia (AI) a zistíme aké má využitie, prínosy a delenie
\item Oboznámime sa s AI online nástrojmi a preskúmame aké nástroje sú dostupné na riešenie nášho problému
\item Preskúmame programovacie a vyberieme jazyky, ktoré budú využité pri experimentoch
\item Vytvoríme experimenty, kde otestujeme všetky požadované parametre 
\item Na záver zanalyzujeme výsledky experimentov, vyvodíme závery a zhodnotíme praktické využitie  prekladu programovacích jazykov pomocou online AI nástrojov. 
\end{itemize}

